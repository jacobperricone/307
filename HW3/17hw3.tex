\documentclass[12pt,letterpaper]{article}
\usepackage{latexsym}
\usepackage[english, activeacute]{babel}
\usepackage{amssymb,amsmath,amsthm}
%\usepackage{epsf}
\usepackage{epsfig}

%\usepackage[active]{srcltx}
%\usepackage{fullpage}
%\usepackage[dvips]{graphicx}
%\DeclareGraphicsExtensions{.eps} \graphicspath{{images/}}
\linespread{1.3}
\parskip 1ex  \parindent 0ex
%\voffset -15mm
\oddsidemargin -0mm \topmargin 0mm \headheight 0pt \headsep 0pt
\textwidth 6.5in \textheight 9in %\marginparsep 0pt \marginparwidth 0pt
\renewcommand{\baselinestretch}{1.3}
\newtheorem{prop}{Proposition}
\newtheorem{Cor}{Corollary}
\newtheorem{lema}{Lemma}
\renewcommand\a{{\bf a}}
\renewcommand\b{{\bf b}}

\newcommand\x{{\bf x}}

\begin{document}
\pagestyle{empty}
\def \NATUR{ I \hspace*{-0.8ex} N}
\def \REALES{ I \hspace*{-0.8ex} R}
\renewcommand{\labelenumi}{\alph{enumi})}
\renewcommand{\labelenumii}{\roman{enumii})}
\newcommand*{\dprime}{^{\prime\prime}\mkern-1.2mu}

%%%%%%%%%%%%%%%%%%%%%%%%%%%%%%%%%%%%%%%%%%%%%%%%%%%%%%%%%%%%
%\hspace*{-4mm} {\includegraphics[bb=0 0 482 723, height=2.3cm]{}}
\hspace{-6mm}
\begin{tabular}{lcr}
CME 307 / MS\&E 311 & \hspace{3in} & Winter 2016-2017 \\ Optimization & & Feb 10, 2017 \\
Prof. Yinyu Ye & & Due Thursday Feb 23 5pm
\end{tabular}

%%%%%%%%%%%%%%%%%%%%%%%%%%%%%%%%%%%%%%%%%%%%%%%%%%%%%%%%%%%%
\bigskip

\begin{center}
{\large \bf Homework Assignment 3 \\  Discuss Session Friday Feb 24 in Class}
\end{center}
%\vfill

%{}

\bigskip

{\textbf{Reading.}} Read selected sections in Luenberger and Ye's {\sl Linear
and Nonlinear Programming Fourth Edition} Chapters 3, 8, 9 and 10.

{\textbf{Solve the following problems:}}
\begin{enumerate}
\item[1.] 8.6 in text book L\&Y.



Consider 
\[
x_{k+1}= \frac{1}{2}[ x_k + \frac{a}{x_k}]
\]
Assuming this sequence converges, what does it converge two and what is the order of convergence. 

\rule{\textwidth}{1pt}

Since the sequence converges we have that $\lim_{k \to \infty} |x_{k+1} - x_{k}| = 0$. Thus we have
\begin{equation*}
\begin{aligned}
\lim_{k \to \infty} |x_{k+1} - x_{k} | &= 0 \\ 
\lim_{k \to \infty}| \frac{1}{2}[ \frac{ a  - x_k^2}{x_k}] | = 0 \\ 
\lim_{k \to \infty}| [ \frac{ a -  x_k^2}{x_k}] | = 0 \\ 
\end{aligned}
\end{equation*}
So the $\lim_{x_k \to \infty} = \pm \sqrt{a}$, where $a > 0$. 

Let $g(x) = \frac{1}{2} x - \frac{a}{2x}$. Then $g(x^{*}) = x^{*}$.  Note
that $g'(x) = \frac{1}{2} - \frac{a}{2x^2}, g\dprime(x) = \frac{1}{x^3}$. Also note that
$g'(x^{*}) = 0$, and $g\prime(x^{*}) = \frac{1}{\pm \sqrt{a}}$, since $x^* = \pm \sqrt{a}$
Taylor expanding $g(x)$

\[
g(x_k) = g(x^{*}) - g'(x^{*})(x_k - x^{*}) + \frac{(x_k - x^*)^2}{2} f\dprime (\psi)
\]

where $\psi$ is in the region $[x_k, x^{*}]$. Now since $g'(x^{*}) = 0$. we have 

\[
g(x_k) = g(x^{*}) +  \frac{(x_k - x^*)^2}{2} f\dprime (\psi)
\]
Letting the successive error $e_k = x_k - x^{*}$. we have

\[
e_{k+1} = g(x_{k+1}) - x^{*} =  \frac{(e_k^2)^2}{2} f\dprime (\psi)
\]



Each successive error term is proportional to the square of the previous error, thus the method has quadratic convergence.

\rule{\textwidth}{1pt}


\item[2.] Prove Lemma 1 of Lecture Note 11. (Hint: using the Taylor Expansion Theorem.)


I am going to prove this slightly differently than the typical taylor expansion proof. First note that if $f(x)$ is $\beta$-Lipschitz, we have

\rule{\textwidth}{1pt}
\[
|| \nabla f(x) - \nabla f(y) || \leq \beta ||x - y ||
\]
For $\beta > 0$. Now let $z(t) = f(y + t(x -y))$ so that $z(0) = f(y)$ and $z(1) = f(x)$. Furthermore note that
\[
g'(t)|_{t = 0 } = \nabla f(y + t(x - y))^{T} (x - y)
\]

and that $\int_{0}^{1} g'(t) dt = g(1) - g(0)$. Then it follows that 
\begin{equation*}
\begin{aligned}
|f(x) - f(y) - \nabla f(y)^T (x - y) | &= | \int_{0}^{1} g'(t) dt - \nabla f(y)^T (x -y)|   \\ 
&= | \int_{0}^{1} \nabla f(y + t(x - y))^T (x - y) dt - \nabla f(y)^T (x - y) | \\ 
&\leq \int_{0}^1 | \nabla f(y + t(x - y))^T (x - y) - \nabla f(y)^T (x - y) |dt  \\ 
&\leq \int_{0}^1 || \nabla f(y + t(x - y)) - \nabla f(y) || ||x - y|| dt   \mbox{    by Cauchy Schwartz } \\ 
&\leq \int_{0}^1 \beta t ||x - y||^2 dt   \mbox{    by $\beta$- Lipschitz } \\ 
&\leq  \beta  \frac{||x - y||^2}{2} dt  
\end{aligned}
\end{equation*}

and thus the proof is complete. 

\rule{\textwidth}{1pt}
\item[3.] Using the simplex method to solve 3.10 in text book L\&Y, assuming that you start from corner origin (there is two typos in
the text book where: ``minimize'' should be :maximize'' and ``$x_1\ge 0$'' should be ``$x_i\ge 0$''.


\rule{\textwidth}{1pt}


We have the problem 

\begin{equation*}
\begin{aligned}
\mbox{maximize } &2x_1 + 4x_2+ x_3 + x_4 \\ 
\mbox{ such that: } & x_1 + 3x_2 + x_4 \leq 4 \\ 
& 2x_1 + x_2 \leq 3 \\ 
& x_2 + 4x_3 + x_4 \leq 3 \\ 
& x_i \geq 0 
\end{aligned}
\end{equation*}

Introducing the slack variables $w_1,w_2, w_3$ we have

\begin{equation*}
\begin{aligned}
\mbox{maximize } &2x_1 + 4x_2+ x_3 + x_4 \\ 
\mbox{ such that: } & \\ 
w_1 &= 4 - x_1 - 3 x_2 - x_4 \\ 
w_2 &= 3 - 2x_1 - x_2 \\
w_3 &= 3 - x_2 - 4x_3 - x_4  \\ 
x_i, w_i \geq 0
\end{aligned}
\end{equation*}

Now not our basic set $B = \{ w_1, w_2, w_3 \}$ is feasible. Noticing that the objective increases most rapidly with $x_2$, we pivot $x_2$ into the basis and $w_1$ out of the basis, yielding $B = \{ x_2, w_2, w_3 \}, NB = \{w_1, x_1, x_3, x_4 \}$.
\begin{equation*}
\begin{aligned}
\mbox{maximize } & \frac{16}{3}  + \frac{1}{3} x_1 - \frac{4}{3}w_1+ x_3 - \frac{1}{3} x_4 \\ 
\mbox{ such that: } & \\ 
x_2 &= \frac{1}{3}[ 4 -  x_1 -  w_1 -  x_4] \\ 
w_2 &= \frac{1}{3} [ 5 - 5x_1  + w_1  + x_4]  \\ 
w_3 &= \frac{1}{3} [ 5 + x_1 + w_1  - 12 x_3 - 2 w_4]  \\ 
x_i, w_i \geq 0
\end{aligned}
\end{equation*}

Now pivoting $x_3$ into the basis and $w_3$ out of the basis $ B = \{x_2, w_2, w_3 \}, NB = \{x_1, x_3, x_4, w_1 \}$. 

\begin{equation*}
\begin{aligned}
\mbox{maximize } & 5.75  + .75 x_1 - \frac{5}{4}w_1 - .25 w_3 - .5 x_4 \\ 
\mbox{ such that: } & \\ 
x_2 &= \frac{1}{3}[ 4 -  x_1 -  w_1 -  x_4] \\ 
w_2 &= \frac{1}{3} [ 5 - 5x_1  + w_1  + x_4]  \\ 
x_3 &= \frac{1}{12} [ 5 + x_1 + w_1  - .25 w_3 -  2 w_4]  \\ 
x_i, w_i \geq 0
\end{aligned}
\end{equation*}

Now, pivoting $x_1$ into the basis and $w_2$ out of the basis so $B= \{x_1, x_2, x_3 \}, NB = \{w_1, w_2, w_3, x_4\}$
\begin{equation*}
\begin{aligned}
\mbox{maximize } & 6.5  - 4.5 w_1 - 1.1w_1 - .25 w_3 - .35 x_4 \\ 
\mbox{ such that: } & \\ 
x_2 &= \frac{1}{10}[ 10 + 2 w_2 -  4 w_1 -  x_4] \\ 
x_1 &= \frac{1}{10} [10  - 6 w_2 + 2 w_1  + 2x_4]  \\ 
x_3 &= \frac{1}{20} [10 - w_2 + 2 w_1  - 5 w_3 -  3 w_4]  \\ 
x_i, w_i \geq 0
\end{aligned}
\end{equation*}

This is optimal, so the optimal value is $6.5$, with $x_1 = 1, x_2 = 1, x_3 = .5, x_4 = 0$. 



\rule{\textwidth}{1pt}





\item[4.] In Logistic Regression, we like to determine $x_0$ and $\x$ to maximize
\[
\left(\prod_{i,c_i=1}\frac{1}{1+\exp(-\a_i^T\x-x_0)}\right)
\left(\prod_{i,c_i=-1}\frac{1}{1+\exp(\a_i^T\x+x_0)}\right).
\]
which is equivalent to maximize the log-likelihood probability
\[
-\sum_{i,c_i=1}\log\left(1+\exp(-\a_i^T\x-x_0)\right)-\sum_{i,c_i=-1}\log\left(1+\exp(\a_i^T\x+x_0)\right).
\]
Or to minimize the log-logistic-loss
\[
\sum_{i,c_i=1}\log\left(1+\exp(-\a_i^T\x-x_0)\right)+\sum_{i,c_i=-1}\log\left(1+\exp(\a_i^T\x+x_0)\right).
\]
\begin{enumerate}
\item[(a)] Write down the gradient vector function of the log-logistic-loss function.

\rule{\textwidth}{1pt}

Let $f(x,x_0)$ be the log-logistic loss function. 
\begin{equation*}
\begin{aligned}
\nabla f(x,x_0)_{x_j} &= \sum_{i,c_i = 1} \frac{-a_{ij} \exp[ -\a_i^T x - x_0 ]}{ 1 + \exp[ -\a_i^T x - x_0 ]} + \sum_{i, c_i = -1} \frac{a_{ij} \exp[ \a_i^T x + x_0 ]}{ 1 + \exp[ \a_i^T x + x_0 ]}  \mbox{  } \forall j \\
\nabla f(x,x_0)_{x_0} &= \sum_{i,c_i = 1} \frac{- \exp[ -\a_i^T x - x_0 ]}{ 1 + \exp[ -\a_i^T x - x_0 ]} + \sum_{i, c_i = -1} \frac{ \exp[ \a_i^T x + x_0 ]}{ 1 + \exp[ \a_i^T x + x_0 ]} \\  
\end{aligned}
\end{equation*}


\rule{\textwidth}{1pt}

\item[(b)] Consider the specific problem described in question (b) of Problem 6 in HW2, and find the solution by
Logistic regression using the KKT conditions. 


\end{enumerate}

\item[5] (Computation Team Work) Use the steepest descent, the accelerated steepest descent, the conjugate direction, and the BB methods to numerically solve the problem in (4.b) and compare their convergence speeds.

\item[6.] Questions (a) to (c) of 8.24 in the text book L\&Y. (Bonus Question: In this method, we fist take the standard $1/\beta$ step-size. If some coordinate in the new iterate become negative, then we take a smaller step-size such that the iterate remain non-negative.  What is the convergence speed of the method?)




Consider a problem of the form 

\begin{equation*}
\begin{aligned}
\mbox{minimize } & f(x) \\ 
\mbox{such that } & x \geq 0
\end{aligned}
\end{equation*}

At a give point $x = (x_1, \hdots, x_n).$ the direction $d = (d_1, \hdots, d_n)$ is determined from the gradient vector $\nabla f(x)^T = g = (g_1, \hdots, g_n)$ by 

\begin{gather*}
d_i = \begin{cases} -g_i \mbox{ if } x_i > 0 \mbox{ or} g_i < 0 \\ 
0 \mbox {if } x_i = 0 \mbox{ and } g_i \geq 0 \end{cases}
\end{gather*}

\begin{enumerate}

\item  What are the first order conditions for a minimum of the problem. 


\rule{\textwidth}{1pt}


The lagrangian of the problem is 

\[
L(x, \lambda) = f(x) - \lambda^T x 
\]
Thus the KKT conditions are

\begin{equation*}
\begin{aligned}
(1) & \nabla f(x) -  \lambda = 0 \\ 
(2) & \lambda_i  \geq 0 \mbox{    } \forall i \\ 
(3) & \lambda_i x_i = 0 \mbox{   } \forall i 
\end{aligned}
\end{equation*}

\rule{\textwidth}{1pt}


\item Show that d, as determined by the algorithm, is zero only at a point that satisfies the first order conditions. 

We have that $d_i = 0$ when $x_i = 0$ and $g_i \geq 0$. From the KKT conditions we know that if $x_i = 0$ for all $i$, we have condition (3) satisfied no matter $\lambda$. Furthermore, since $g_i \geq 0$, we have that $\nabla f(x) = \lambda  \geq 0, \forall i$. Thus condition (1) and (2), are satisfied. Therefore if $d_i = 0$, the KKT conditions are satisfied.  Now $d_i \neq 0$, then  $x_i \neq 0$ and $g_i < 0$ it is obvious the KKT conditions do not hold. Assume $d_i < 0$ and $x_i \neq 0$, then we have that $\nabla f(x)_i < 0$, since $\nabla f(x)_i = \lambda_i$, we must have that $\lambda_i < 0$, which violates the second KKT condition. Futhermore since $\lambda_i x_i = 0$, for condition (3), it follows that $\lambda_i$ must equal zero, which is a contradiction. 

\item Show that if $d \neq 0$, it is possible to decrease the value of f by movement along d. 







\end{enumerate}



\item[7.] Consider the problem 8.24 in the text L\&Y, but solve the Barrier problem
\[\min_{x>0}\ \phi(\x):=f(\x)-\mu\sum_{j=1}^n\ln(x_j)\]
where $\mu$ is a small positive constant.
\begin{itemize}
\item[(a)] What are the first-order necessary conditions for the minimizer of $\phi(\x)$?

\item[(b)] Let $f(\x)=\frac{1}{2}\|A\x-\b\|^2$. Then develop the standard steepest gradient iterative process to find a KKT solution.

\item[(c)] Again let $f(\x)=\frac{1}{2}\|A\x-\b\|^2$, and at the $k$ step consider the (affine) scaled gradient vector
\[(X^k)^2\nabla\phi(\x^k),\]
where $X^k$ is the diagonal matrix of $k$th iterate $\x^k(>0)$. Let us generate the next iterate by
\[\x^{k+1}=\x^k-\alpha^k(X^k)^2\nabla\phi(\x^k),\]
for some stepsize $\alpha^k$. How is this process different from (b)

\item[(d)] (Computation Team Work) Generate some random data and solve them with $\mu=0$ and $\mu=10^{-4}$, and observe any possible difference between methods (b) and (c).
\end{itemize}

\item[8.] Now consider the problem:
\[\min\ f(X)\quad \mbox{s.t.}\quad X\succeq 0,\]
and its Barrier problem
\[\min_{X\succ 0}\ \phi(X):=f(X)-\mu\ln\det(X)\]
where $\mu$ is a small positive constant.

\begin{itemize}
\item[(a)] What are the (symmetric) gradient matrix and the first-order necessary conditions for the minimizer of $\phi(X)$?

\item[(b)] Let $f(X)=\frac{1}{2}\|{\cal A}X-\b\|^2$, where
\[{\cal A}X=\left(\begin{array}{c}
                           A_1\bullet X\\
                           ...\\
                           A_m\bullet X
                           \end{array}\right).\]
for some symmetric data matrices $A_1$,...,$A_m$, and data vector $\b$.
Then develop the standard steepest gradient iterative process to find a KKT solution.

\item[(c)] Again let $f(X)=\frac{1}{2}\|{\cal A}X-\b\|^2$, and at the $k$ step consider the scaled gradient matrix
\[X^k\nabla\phi(X^k)X^k,\]
where $X^k\succ 0$ is the $k$th iterate solution matrix. Let us generate the next iterate by
\[X^{k+1}=X^k-\alpha^kX^k\nabla\phi(X^k)X^k,\]
for some stepsize $\alpha^k$. What is the difference between the two iterative processes in (b) and (c).

\item[(d)] (Computation Team Work) Apply this method for Sensor Network Localization problems in HW1 and HW2 varying
with $\mu=0$ to $\mu=10^{-4}$.
\end{itemize}

\item[9.] (20pts Computation Team Work) 
All pieces of Computation Team Work stated above, and one more decribed below.

There is a simple nonlinear least squares approach for Sensor Network Localization:
\begin{equation}\label{snl-nls}
\begin{array}{lll}
&\min        & \sum_{(ij)\in N_x}\left(\|\x_i-\x_j\|^2 - d_{ij}^2\right)^2+\sum_{(kj)\in N_a} \left(\|\a_k-\x_j\|^2 - d_{kj}^2\right)^2
\end{array}
\end{equation}
which is an unconstrained nonlinear minimization problem.

Use the SDP solution (which may not accurate) as the initial solution for (\ref{snl-nls}) and apply the Steepest Descent Method for a few number of steps for the two SNL problems in HW1 and HW2, where you can add some noises to the distance measurements. How is the final solution come out after steepest descent? How does the final solution compare with the SDM starting from a random initial solution?

\end{enumerate}

\end{document}
